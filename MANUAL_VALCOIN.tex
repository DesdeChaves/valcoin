\hypertarget{manual-de-utilizador-avanuxe7ado-da-aplicauxe7uxe3o-valcoin}{%
\section{Manual de Utilizador Avançado da Aplicação
Valcoin}\label{manual-de-utilizador-avanuxe7ado-da-aplicauxe7uxe3o-valcoin}}

Este manual descreve em detalhe as funcionalidades da plataforma
Valcoin, um ecossistema gamificado de literacia financeira. Abrange os
perfis de Administrador, Professor e Aluno (Valkid/Valteen).

\begin{center}\rule{0.5\linewidth}{0.5pt}\end{center}

\hypertarget{administrador}{%
\subsection{1. Administrador}\label{administrador}}

O Administrador detém o controlo total sobre o ecossistema Valcoin,
sendo responsável pela sua configuração, personalização e supervisão
económica.

\hypertarget{dashboard-de-controlo}{%
\subsubsection{1.1. Dashboard de Controlo}\label{dashboard-de-controlo}}

O dashboard é o centro nevrálgico da plataforma, apresentando métricas
vitais em tempo real: - \textbf{Métricas de Utilizadores:} Total de
utilizadores, utilizadores ativos, e a sua distribuição (Aluno,
Professor, Admin). - \textbf{Métricas de Transações:} Volume total de
transações, estado (concluídas, pendentes, rejeitadas) e tipo
(crédito/débito). - \textbf{Métricas Económicas:} - \textbf{Total de
Valcoins (VC):} Soma dos VC em todas as carteiras e os VC gastos. -
\textbf{Taxa de Câmbio:} Valor de 1 VC em Euros, tanto a taxa fixa como
uma taxa dinâmica calculada com base nas receitas da escola. -
\textbf{Receitas da Escola:} Total de receitas registadas no sistema. -
\textbf{Métricas de Legado:} Pontos de legado totais e crescimento
mensal, refletindo as contribuições mais valiosas dos alunos.

\hypertarget{configurauxe7uxe3o-da-economia}{%
\subsubsection{1.2. Configuração da
Economia}\label{configurauxe7uxe3o-da-economia}}

A principal responsabilidade do Admin é modelar a economia da Valcoin.

\hypertarget{gestuxe3o-de-regras-de-transauxe7uxe3o}{%
\paragraph{1.2.1. Gestão de Regras de
Transação}\label{gestuxe3o-de-regras-de-transauxe7uxe3o}}

As ``Transaction Rules'' são o motor da economia. O Admin pode criar,
editar e desativar regras que definem como os Valcoins são ganhos e
gastos. Cada regra pode ter: - \textbf{Nome e Valor:} Ex: ``Participação
na Aula'', +10 VC. - \textbf{Permissões:} Quem pode iniciar e quem pode
receber a transação (e.g., Professor -\textgreater{} Aluno). -
\textbf{Limites:} Restrições de utilização por período (mensal, anual) e
por disciplina. - \textbf{Categoria:} Uma categoria especial é
\textbf{``Legado''}. Transações com esta categoria são registadas
permanentemente no perfil do aluno, significando uma contribuição de
grande valor. - \textbf{Requisitos:} Pode exigir um ano de escolaridade
mínimo/máximo para a sua aplicação.

\hypertarget{produtos-financeiros}{%
\paragraph{1.2.2. Produtos Financeiros}\label{produtos-financeiros}}

O Admin cria os produtos financeiros que os alunos (Valteens) podem
subscrever: - \textbf{Produtos de Poupança:} Define contas poupança com
prazo (em meses), taxa de juro, e depósitos mínimos/máximos. -
\textbf{Produtos de Crédito:} Define as condições dos micro-empréstimos,
como o montante máximo, prazo de pagamento e taxas de juro associadas.

\hypertarget{configurauxe7uxf5es-gerais-do-sistema}{%
\paragraph{1.2.3. Configurações Gerais do
Sistema}\label{configurauxe7uxf5es-gerais-do-sistema}}

\begin{itemize}
\tightlist
\item
  \textbf{Utilizadores Especiais:} Designar contas de sistema cruciais,
  como o \texttt{ivaDestinationUserId} (para onde vai o IVA das
  transações) e o \texttt{interestSourceUserId} (a conta que financia os
  juros das poupanças).
\end{itemize}

\hypertarget{gestuxe3o-fiscal-e-de-produtos}{%
\paragraph{1.2.4. Gestão Fiscal e de
Produtos}\label{gestuxe3o-fiscal-e-de-produtos}}

Para simular uma economia real, o Admin gere um sistema fiscal e de
categorização de produtos: - \textbf{Taxas de IVA:} O sistema permite a
configuração de múltiplas taxas de IVA (e.g., Normal, Reduzida, Isenta).
Cada produto ou serviço na loja é associado a uma destas taxas. -
\textbf{Categorias de Produtos:} O Admin pode criar categorias para os
produtos da loja (e.g., ``Material Escolar'', ``Alimentação'',
``Bilhetes''). Estas categorias têm um propósito de literacia fiscal: -
\textbf{Dedução Fiscal:} Uma categoria pode ser marcada como
\texttt{is\_deductible}, permitindo que uma parte ou a totalidade do
valor gasto em produtos dessa categoria possa ser ``deduzida'' da coleta
de impostos, ensinando na prática como funcionam os benefícios fiscais.

\hypertarget{gestuxe3o-da-comunidade}{%
\subsubsection{1.3. Gestão da
Comunidade}\label{gestuxe3o-da-comunidade}}

\hypertarget{gestuxe3o-de-casas-houses}{%
\paragraph{1.3.1. Gestão de Casas
(Houses)}\label{gestuxe3o-de-casas-houses}}

As ``Houses'' são comunidades dentro da escola (semelhante às casas de
Hogwarts), fomentando a colaboração e competição saudável. -
\textbf{Criação:} O Admin cria as casas, definindo nome, cor, logo e
descrição. - \textbf{Atribuição:} Associa um Professor como ``Chefe de
Casa'' e pode nomear um Aluno como ``Líder''. - \textbf{Gestão de
Membros:} Adiciona e remove alunos das casas. - \textbf{Supervisão:} O
dashboard de cada casa mostra estatísticas agregadas: saldo total dos
membros, dívida total, e a percentagem de membros com contas poupança
ativas.

\hypertarget{gestuxe3o-de-utilizadores-e-matruxedculas}{%
\paragraph{1.3.2. Gestão de Utilizadores e
Matrículas}\label{gestuxe3o-de-utilizadores-e-matruxedculas}}

\begin{itemize}
\tightlist
\item
  \textbf{Importação em Massa:} Para além da criação manual, o Admin
  pode carregar ficheiros (e.g., CSV) para criar em massa utilizadores
  (alunos, professores), disciplinas, e realizar matrículas em turmas e
  inscrições em disciplinas, utilizando listas de serviços letivos da
  escola.
\item
  \textbf{Aprovação de Empréstimos:} Avalia e aprova ou rejeita os
  pedidos de micro-empréstimos feitos pelos alunos.
\end{itemize}

\begin{center}\rule{0.5\linewidth}{0.5pt}\end{center}

\hypertarget{professor}{%
\subsection{2. Professor}\label{professor}}

O Professor é o principal agente de dinamização da economia Valcoin no
dia-a-dia, utilizando as regras criadas pelo Admin para interagir com os
alunos.

\hypertarget{principais-funcionalidades}{%
\subsubsection{Principais
Funcionalidades}\label{principais-funcionalidades}}

\begin{itemize}
\tightlist
\item
  \textbf{Atribuir Valcoins:} Utiliza as ``Transaction Rules''
  pré-definidas para atribuir Valcoins aos alunos.
\item
  \textbf{Atribuir Pontos de Legado:} Ao usar uma regra da categoria
  ``Legado'', o professor confere uma honra especial ao aluno.
\item
  \textbf{Gestão de Turmas:} Visualiza os alunos das suas turmas e o seu
  progresso individual.
\item
  \textbf{Chefe de Casa (se aplicável):} Se for designado Chefe de uma
  Casa, tem acesso ao dashboard dessa casa.
\item
  \textbf{Acompanhamento:} Pode consultar o histórico de transações, o
  saldo, as poupanças e os empréstimos dos seus alunos.
\item
  \textbf{Validação de Bilhetes:} Pode usar a interface de validação
  para marcar bilhetes de eventos como ``utilizados''.
\end{itemize}

\begin{center}\rule{0.5\linewidth}{0.5pt}\end{center}

\hypertarget{aluno}{%
\subsection{3. Aluno}\label{aluno}}

O Aluno interage com o ecossistema Valcoin para aprender a gerir as suas
finanças pessoais de forma prática e motivadora.

\hypertarget{valkid-alunos-mais-novos}{%
\subsubsection{3.1. Valkid (Alunos mais
novos)}\label{valkid-alunos-mais-novos}}

A interface é simples e visual, focada em conceitos básicos: consulta de
saldo, histórico de conquistas e uma loja de recompensas simples.

\hypertarget{valteen-alunos-mais-velhos}{%
\subsubsection{3.2. Valteen (Alunos mais
velhos)}\label{valteen-alunos-mais-velhos}}

A plataforma oferece um conjunto robusto de ferramentas de literacia
financeira.

\hypertarget{a-tua-conta}{%
\paragraph{3.2.1. A Tua Conta}\label{a-tua-conta}}

\begin{itemize}
\tightlist
\item
  \textbf{Dashboard Pessoal:} Um resumo da sua vida financeira: saldo,
  histórico, poupanças e empréstimos.
\item
  \textbf{O Teu Legado:} Uma secção especial no perfil que exibe todas
  as distinções da categoria ``Legado'' que recebeu.
\end{itemize}

\hypertarget{funcionalidades-financeiras}{%
\paragraph{3.2.2. Funcionalidades
Financeiras}\label{funcionalidades-financeiras}}

\begin{itemize}
\tightlist
\item
  \textbf{Contas Poupança:} Pode subscrever produtos de poupança,
  depositar Valcoins e acompanhar o crescimento com juros.
\item
  \textbf{Micro-Empréstimos:} Pode solicitar um empréstimo, justificando
  o pedido, e tem a responsabilidade de o pagar.
\end{itemize}

\hypertarget{vida-em-comunidade}{%
\paragraph{3.2.3. Vida em Comunidade}\label{vida-em-comunidade}}

\begin{itemize}
\tightlist
\item
  \textbf{A Minha Casa (House):} Pertence a uma casa, colabora com
  colegas e acompanha o ranking e estatísticas do grupo.
\end{itemize}

\begin{center}\rule{0.5\linewidth}{0.5pt}\end{center}

\hypertarget{a-loja-valcoin-aprendizagem-e-funcionamento}{%
\subsection{4. A Loja Valcoin: Aprendizagem e
Funcionamento}\label{a-loja-valcoin-aprendizagem-e-funcionamento}}

A loja é mais do que um mercado; é uma ferramenta pedagógica central
para a literacia financeira.

\hypertarget{objetivos-pedaguxf3gicos}{%
\subsubsection{4.1. Objetivos
Pedagógicos}\label{objetivos-pedaguxf3gicos}}

\begin{itemize}
\tightlist
\item
  \textbf{Valor vs.~Preço:} Os alunos aprendem que os Valcoins têm um
  valor real (convertível para Euros no dashboard do Admin) e que os
  produtos têm um custo que reflete esse valor, acrescido de impostos
  (IVA).
\item
  \textbf{Tomada de Decisão Financeira:} A loja força o aluno a decidir
  entre a gratificação imediata (comprar produtos mais baratos) e o
  planeamento a longo prazo (poupar para um item mais caro e desejado).
\item
  \textbf{Orçamentação:} A necessidade de gerir um saldo limitado para
  adquirir bens ensina a importância de criar um orçamento e de fazer
  escolhas.
\item
  \textbf{Impacto Fiscal:} Ao comprar produtos de diferentes categorias
  e com diferentes taxas de IVA, os alunos observam na prática como os
  impostos afetam o custo final de um produto e como certas despesas (de
  categorias dedutíveis) podem ter benefícios.
\end{itemize}

\hypertarget{funcionamento}{%
\subsubsection{4.2. Funcionamento}\label{funcionamento}}

\begin{itemize}
\tightlist
\item
  \textbf{Criação de Produtos:} O Admin (ou vendedores designados)
  adiciona produtos à loja, definindo nome, preço em Valcoins,
  quantidade em stock, categoria e a taxa de IVA aplicável.
\item
  \textbf{Compra pelo Aluno:} O aluno navega na loja, escolhe um produto
  e confirma a compra. O sistema debita automaticamente o valor total
  (preço + IVA) do seu saldo.
\item
  \textbf{Gestão de Stock:} O sistema gere o stock em tempo real. Se um
  produto esgota, fica indisponível para compra.
\end{itemize}

\begin{center}\rule{0.5\linewidth}{0.5pt}\end{center}

\hypertarget{sistema-de-bilhuxe9tica-digital}{%
\subsection{5. Sistema de Bilhética
Digital}\label{sistema-de-bilhuxe9tica-digital}}

Integrado na loja, o sistema de bilhética permite a venda e gestão de
entradas para eventos (e.g., festas da escola, cinema, workshops) de
forma segura e digital.

\hypertarget{criauxe7uxe3o-e-compra}{%
\subsubsection{5.1. Criação e Compra}\label{criauxe7uxe3o-e-compra}}

\begin{itemize}
\tightlist
\item
  \textbf{Produto-Bilhete:} O Admin cria um produto na loja e marca-o
  como sendo um ``bilhete'' (\texttt{is\_ticket}).
\item
  \textbf{Compra Segura:} Quando um aluno compra este produto, o sistema
  gera uma entrada única e pessoal na base de dados -- o bilhete
  digital.
\end{itemize}

\hypertarget{utilizauxe7uxe3o-e-validauxe7uxe3o}{%
\subsubsection{5.2. Utilização e
Validação}\label{utilizauxe7uxe3o-e-validauxe7uxe3o}}

\begin{itemize}
\tightlist
\item
  \textbf{Download do Bilhete:} O aluno pode descarregar um ficheiro PDF
  que contém os detalhes do evento e um QR Code único.
\item
  \textbf{Validação na Entrada:} No dia do evento, um responsável (e.g.,
  um professor, um segurança) utiliza uma interface de validação
  especial:

  \begin{enumerate}
  \def\labelenumi{\arabic{enumi}.}
  \tightlist
  \item
    Acede à página de validação (que pode ser pública).
  \item
    Escaneia o QR Code do bilhete do aluno ou digita o seu ID.
  \item
    O sistema verifica em tempo real se o bilhete é válido e se já foi
    utilizado.
  \end{enumerate}
\item
  \textbf{Marcação como ``Utilizado'':} Após a validação bem-sucedida, o
  responsável marca o bilhete como ``utilizado'', o que invalida o QR
  Code e impede a sua reutilização, garantindo que cada bilhete é usado
  apenas uma vez.
\item
  \textbf{Gestão pelo Admin/Vendedor:} O criador do evento pode aceder a
  um painel para ver todos os bilhetes vendidos, quem os comprou e o seu
  estado (válido ou utilizado).
\end{itemize}
